\documentclass{article}
\usepackage{graphicx}
\usepackage[margin=3cm]{geometry}
\usepackage{subcaption}

\usepackage[utf8]{inputenc}
\usepackage[T1]{fontenc}
\usepackage[spanish,es-tabla]{babel}

\usepackage{href-ul}
\usepackage{tcolorbox}
\usepackage{listings}
\usepackage{tocloft}
\usepackage[table]{xcolor}
\usepackage{booktabs}

\renewcommand{\contentsname}{Índice}
\renewcommand{\cfttoctitlefont}{\Huge\bfseries}
\renewcommand{\cftdotsep}{1}



\begin{document}

    % ================================= PORTADA =========================================
    \begin{titlepage}
        \centering

        {\Large \textbf{Programación Multimedia y Dispositivos Móviles}\par}
        
        \vspace{2cm}
        {\Huge \textbf{Documentación API dockerizada}\par}
        \vspace{0.5cm}
        {\LARGE Servidor Backend con Express y MongoDB\par}
        
        \vspace{2cm}
        \noindent\rule{10cm}{0.4pt}
        
        \vspace{2cm}
        {\Large \textbf{Autor:}\par}
        \vspace{0.3cm}
        {\large Santi Martínez\par}
        
        \vspace{10cm}
        {\large \today}
        
    \end{titlepage}

    \newpage
    \tableofcontents

    % ================================= INTRODUCCIÓN
    \newpage
    \section{Introducción}

        \subsection{Contexto del proyecto}

            Este proyecto aborda la dockerización de una API REST desarrollada con Node.js y Express, que utiliza MongoDB como sistema de gestión de base de datos. La API implementa funcionalidades de gestión de usuarios y grupos.

            La aplicación se compone de dos servicios principales que deben funcionar de manera coordinada: el servidor de aplicación que expone los endpoints REST y la base de datos MongoDB que persiste la información.
            
            La dockerización de estos componentes permite encapsular cada servicio en contenedores independientes, facilitando su gestión, despliegue y mantenimiento.

        \subsection{Objetivos de la dockerización}

            Los principales objetivos para realizar la dockerización de la API son los siguientes:

            \begin{itemize}
                \item \textbf{Consistencia:} Permite la ejecución del entorno en cualquier ordenador preparado para ejecutar Docker; es decir, funciona de manera autónoma, llevando todo lo necesario en su interior y operando desde ahí, de forma similar a un caballo de Troya.

                \item \textbf{Aislamiento de componentes:} Cada servicio (API y MongoDB) se ejecuta en su propio contenedor con su propio sistema de archivos, procesos y red, evitando conflictos de dependencias.

                \item \textbf{Portabilidad:} Permitir que la aplicación se ejecute de manera consistente en cualquier sistema operativo (Windows, macOS, Linux).

                \item \textbf{Simplificación del despliegue:} Reducir el proceso de instalación y configuración.

                \item \textbf{Escalabilidad:} Permite crear varias copias de un servicio para que se pueda ejecutar en varios lugares al mismo tiempo.
                
                \item \textbf{Gestión de dependencias:} Encapsular todas las dependencias que usa la API (mongoose, express, dotenv, etc ...) de la aplicación dentro de la imagen Docker, garantizando que siempre se mantengan correctas.
            \end{itemize}


        \subsection{Tecnologías utilizadas}

            El conjunto de tecnologías utilizado es el siguiente:

            \begin{itemize}
                \item \textbf{Node.js 20:} Entorno de ejecución de JavaScript del lado del servidor.

                \item \textbf{Express.js:} Framework de Node para la realización de servidores.

                \item \textbf{MongoDB 6:} Base de datos NoSQL orientada a documentos.

                \item \textbf{Mongoose:} ODM (Object Document Mapper) que permite que Node.js se comunique y gestione datos en MongoDB.

                \item \textbf{Docker:} Plataforma de contenedorización que permite empaquetar aplicaciones con todas sus dependencias en contenedores estandarizados. Proporciona un aislamiento ligero y eficiente.

                \item \textbf{Docker Compose:} Herramienta de docker para orquestar varios contenedores a la vez. Permite configurar todos los servicios, redes y volúmenes de la aplicación en un único archivo YML (docker-compose.yml).

                \item \textbf{Variables de entorno:} Variables de configuración del programa, guardadas de forma privada en un .env.
            \end{itemize}


            La combinación de estas tecnologías crea un ecosistema robusto, escalable y fácil de mantener.


    % ================================= FUNDAMENTOS TEÓRICOS
    \newpage
    \section{Fundamentos teóricos}

        \subsection{¿Qué es Docker?}
            Es una plataforma que ejecuta aplicaciones en contenedores, asegurando que su funcionamiento sea el correcto en cualquier sistema.

            \vspace{0.3cm}
            \textbf{Docker corre principalmente sobre Linux}, porque utiliza cacterísticas del kernel de Linux para los contenedores.
            \begin{itemize}
                \item En \textbf{Linux}, se ejecuta de \underline{forma nativa}.
                \item En \textbf{Windows} o \textbf{Mac}, usa una \underline{máquina virtual ligera con Linux} para poder correr contenedores.
            \end{itemize}

        \subsection{Contenedores vs Máquinas Virtuales}

            \begin{table}[h!]
                \centering
                \rowcolors{2}{white}{blue!15}
                \begin{tabular}{@{} l l l @{}}
                    \toprule
                    \textbf{Características} & \textbf{Contenedor} & \textbf{Máquina Virtual (VM)} \\
                    \midrule
                    Sistema operativo & Comparte el SO del host & Cada VM tiene su propio SO completo \\
                    Peso & Ligero, rápido de iniciar & Pesado, tarda más en iniciar \\
                    Recursos & Usa solo lo necesario & Consume más recursos, reserva CPU/RAM \\
                    Aislamiento & Aislado a nivel de procesos & Aislado a nivel de hardware virtual \\
                    Portabilidad & Muy portable & Menos portable \\
                    \bottomrule
                \end{tabular}
                \caption{Comparación entre contenedores y máquinas virtuales}
            \end{table}


        \subsection{Conceptos clave}
        
            Para una mayor comprensión de la utilización de Docker, hay que analizar también unos conceptos principales; estos son de vital importancia:

            \begin{itemize}
                \item \textbf{Imágenes}: Plantilla que contiene todo lo necessario para ejecutar una aplicación perfectamente (Mongo, Ubuntu, Odoo, ...)
                \item \textbf{Contenedores}: Instancia de una imagen, la cual se encuentra aislada.
                \item \textbf{Volúmenes}: Almacenamiento persistente que conserva datos fuera del contenedor. Básicamente, en el caso de eliminar el contenedor por lo que sea, los datos guardados en el volumen siguen existiendo en el host. Se utiliza para datos importantes.
                \item \textbf{Redes}: Medio que permite la comunicación entre contederores y el exterior. Por ejemplo a la hora de realizar manualmente ejecutarse a un contenedor, la parte del comando \texttt{-p 3000:3000}, el primer puerto es del host y el segundo del contenedor, eso hace que ambos se comuniquen.
            \end{itemize}

        \subsection{Docker Compose}

            Docker Compose es una herramienta que permite orquestar varios contenedores mediante un archivo de configuración (\textbf{docker-compose.yml}). Facilita levantar, detener y administrar varios contenedores juntos, incluyendo servicios, redes y volúmenes.


    % ================================= ARQUITECTURA
    \newpage
    \section{Arquitectura}

        \begin{figure}[h!]
            \centering
            \includegraphics[width=0.5\textwidth]{images/API.png}
            \caption{Dockerfile}
        \end{figure}

        \subsection{Descripción de la API}
            
        \subsection{Componentes del sistema}

        \subsection{Diagrama de arquitectura}

        \subsection{Flujo de comunicación}



    % ================================= DOCKERFILE: CONSTRUCCIÓN DE LA IMAGEN
    \newpage
    \section{Construcción de la imagen}

        Es un archivo de texto que contiene instrucciones para construir una imagen de Docker. Define la base del sistema, dependencias, configuraciones y comandos necesarios para que la aplicación se ejecute de forma consistente en cualquier contenedor.

        \subsection{Dockerfile}
            \begin{figure}[h!]
                \centering
                \includegraphics[width=0.5\textwidth]{images/Dockerfile.png}
                \caption{Dockerfile}
            \end{figure}

        \subsection{Análisis del dockerfile}
            \begin{enumerate}
                \item \textbf{FROM node:20}. Indica la imagen base para construir un contenedor. En este Dockerfile se utiliza la version 20, la cual incluye ya Node y npm instalados.
                \item \textbf{WORKDIR /app}. Establece el directorio de trabajo dentro del contenedor en /app. Todos los comandos que vienen a continuación se ejecutarán dentro de esta carpeta.
                \item \textbf{COPY package*.json ./}. Copia los archivos package.json y package-lock.json desde el directorio del host al directorio del contenedor. Así se instalarán las dependencias necesarias en el contenedor.
                \item \textbf{RUN npm install}. Ejecuta npm install dentro del contenedor para instalar las dependencias del proyecto, las cuales se pueden instalar gracias al paso anterior.
                \item \textbf{COPY . .}. Copia todo el código del proyecto al directorio del contenedor
                \item \textbf{EXPOSE 3000}. Sugiere que el contenedor escuche el puerto 3000
                \item \textbf{CMD ["npm","start"]}. Comando que se va a ejecutar cuando se inicia el contenedor, iniciando así la aplicación.
            \end{enumerate}


    
    % ================================= DOCKER COMPSE: ORQUESTACIÓN DE SERVICIOS
    \newpage
    \section{Docker Compose: Orquestación de servicios}

        Es una herramienta que permite definir y ejecutar aplicaciones que usan múltiples contenedores mediante un archivo de configuración (docker-compose.yml). Describe servicios, redes y volúmenes, facilitando desplegar y administrar toda la aplicación de manera consistente.

        \subsection{Estructura del archivo docker-compose.yml}
            \begin{figure}[h!]
                \centering
                \includegraphics[width=0.8\textwidth]{images/docker-compose.png}
                \caption{docker-compose.yml}
            \end{figure}

        \subsection{Servicios}
        % build, container_name, ports, environment, depends_on
            \begin{itemize}
                \item \textbf{build}. Define la ruta o Dockerfile para construir una imagen personalizada del servicio antes de ejecutarlo.
                
                \vspace{0.3cm}
                En este proyecto se utiliza para: \texttt{docker build -t mi\_api:v1.0.0 .}
                
                \item \textbf{image}. Especifica la imagen de Docker que se va a usar para el servicio; puede ser desde Docker Hub o de forma local.
                
                \vspace{0.3cm}
                En este proyecto se hará de forma local.
                
                \item \textbf{container\_name}. Le da un nombre al contenedor, sino lo hace automáticamente.
                
                \item \textbf{ports}. Mapea puertos del contenedor a puertos del host, permitiendo que los servicios sean accesibles desde el exterior.
                
                \item \textbf{environment}. Define variables de entorno que se pasarán al contenedor para configurarlo.
                
                \item \textbf{depends\_on}. Indica que el servicio en el que se sitúa depende de otros, lo que asegura que estos otros se inicien antes.
                
                \item \textbf{volumes}. Permite montar directorios/archivos del host dentro del contenedor, así se persisten los datos y se comparte información entre ambos.
            \end{itemize}

        \subsection{Configuración de volúmenes}
            
            \begin{figure}[h!]
                \centering
                \includegraphics[width=0.4\textwidth]{images/volumes.png}
                \caption{Volumen en el docker-compose.yml}
            \end{figure}

            En el docker-compose.yml se declara el volumen \textbf{mongo\_data}, y este garantiza que los datos de Mongo persistan y no se pierdan al eliminar o reiniciar el contenedor.

        \subsection{Variables de entorno}

            Variables de configuración del programa, guardadas de forma privada en un .env.

            Posteriormente el .env en el que se encuentran estas variables será guardado en .gitignore y en .dockerignore.

            \begin{tcolorbox}[colback=blue!3!white,colframe=blue!60!black,fonttitle=\bfseries]
                \textbf{.dockerignore} y \textbf{.gitignore} son archivos en los que se introduce todo el contenido que no debe ser subido por seguridad u optimización.
                
                \vspace{0.3cm}
                \underline{Ejemplo}: .env por seguridad y node\_modules porque pesan mucho y no es necesaria su carga.
            \end{tcolorbox}

        \subsection{Dependencias entre servicios}
            
            \noindent
            \begin{minipage}[t]{0.50\textwidth}
                \vspace{0pt}
                En el \textbf{docker-compose.yml}, el servicio \textbf{api} depende del servicio \textbf{mongo}, lo que significa que Docker Compose inicia primero el contenedor de Mongo antes de levantar la API. Así se asegura que la base de datos esté en ejecución cuando la API intente conectarse y no se quede colgada.

                \vspace{0.5cm}
                El \textbf{depends\_on} garantiza el orden de inicio de los contenedores.
            \end{minipage}
            \hspace{1cm}
            \begin{minipage}[t]{0.4\textwidth}
                \vspace{0pt}
                \centering
                \includegraphics[width=1\textwidth]{images/depends-on.png}
            \end{minipage}


    % ================================= PROCESO DE DOCKERIZACIÓN
    \newpage
    \section{Proceso de dockerización}

        \subsection{Preparación del entorno}

        \subsection{Construcción de la imagen}

        \subsection{Creación y ejecución de contenedores}

        \subsection{Verificación del funcionamiento}


    % ================================= COMANDOS DOCKER ÚTILES
    \newpage
    \section{Comandos Docker útiles}

        \subsection{Comandos básicos}

        \subsection{Gestión de imágenes}

        \subsection{Gestión de contenedores}

        \subsection{Docker Compose CLI}

        \subsection{Comandos de depuración}


    % ================================= REDES Y COMUNICACIÓN
    \newpage
    \section{Redes y comunicación}

        \subsection{Red por defecto de Docker Compose}

        \subsection{Resolución de nombres DNS}

        \subsection{Comunicación entre contenedores}


    % ================================= PERISISTENCIA DE DATOS
    \newpage
    \section{Persistencia de datos}

        \subsection{Volúmenes en Docker}

        \subsection{Volumen mongo\_data}

        \subsection{Backup y recuperación}


    % ================================= PRUEBAS Y VALIDACIÓN
    \newpage
    \section{Pruebas y validación}

        \subsection{Verificación de contenedores activos}

        \subsection{Pruebas de conectividad}

        \subsection{Pruebas de endpoints}

        \subsection{Logs y monitorización}
        

    % ================================= OPTIMIZACIONES Y MEJORAS
    \newpage
    \section{Optimizaciones y mejoras}

        \subsection{Optimización del Dockerfile}

        \subsection{Multi-stage builds}

        \subsection{Reducción del tamaño de la imagen}

        \subsection{Cache de capas}


    % ================================= SEGUIRDAD
    \newpage
    \section{Seguridad}

        \subsection{Gestión de secretos}

        \subsection{Usuarios no privilegiados}

        \subsection{Escaneo de vulnerabilidades}

        \subsection{Actualización de imágenes base}


    % ================================= DESPLIEGUE EN PRODUCCIÓN
    \newpage
    \section{Despliegue en producción}

        \subsection{Diferencias con desarrollo}

        \subsection{Configuración para producción}

        \subsection{Escalabilidad}

        \subsection{Herramientas de orquestación}


    % ================================= TROUBLESHOOTING
    \newpage
    \section{Troubleshooting}

        \subsection{Problemas comunes}

        \subsection{Errores de conexión}

        \subsection{Problemas de permisos}

        \subsection{Debugging de contenedores}


    % ================================= CONCLUSIONES
    \newpage
    \section{Conclusiones}

        \subsection{Ventajas de la dockerización}

        \subsection{Resultados obtenidos}

        \subsection{Trabajo futuro}


    % ================================= REFERENCIAS
    \newpage
    \section{Referencias}

        \subsection{Documentación oficial}

        \subsection{Recursos adicionales}


\end{document}